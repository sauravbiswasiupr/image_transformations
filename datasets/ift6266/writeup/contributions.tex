\documentclass{article} % For LaTeX2e
\usepackage{times}
\usepackage{wrapfig}
\usepackage{amsthm,amsmath,bbm} 
\usepackage[psamsfonts]{amssymb}
\usepackage{algorithm,algorithmic}
\usepackage[utf8]{inputenc}
\usepackage{graphicx,subfigure}
\usepackage[numbers]{natbib}

\addtolength{\textwidth}{10mm}
\addtolength{\evensidemargin}{-5mm}
\addtolength{\oddsidemargin}{-5mm}

%\setlength\parindent{0mm}

\begin{document}

\begin{center}
{\Large Deep Self-Taught Learning for Handwritten Character Recognition}

{\bf \large Information on Main Contributions}
\end{center}

\setlength{\parindent}{0cm}

%\vspace*{-2mm}
\section*{Background and Related Contributions}
%\vspace*{-2mm}
%{\large \bf Background and Related Contributions}

Recent theoretical and empirical work in statistical machine learning has
demonstrated the potential of learning algorithms for {\bf deep
  architectures}, i.e., function classes obtained by composing multiple
levels of representation
\citep{Hinton06,ranzato-07-small,Bengio-nips-2006,VincentPLarochelleH2008,ranzato-08,Larochelle-jmlr-2009,Salakhutdinov+Hinton-2009,HonglakL2009,HonglakLNIPS2009,Jarrett-ICCV2009,Taylor-cvpr-2010}.
See~\citet{Bengio-2009} for a review of deep learning algorithms.

{\bf Self-taught learning}~\citep{RainaR2007} is a paradigm that combines
principles of semi-supervised and multi-task learning: the learner can
exploit examples that are unlabeled and possibly come from a distribution
different from the target distribution, e.g., from other classes than those
of interest.  Self-taught learning has already been applied to deep
learners, but mostly to show the advantage of unlabeled
examples~\citep{Bengio-2009,WestonJ2008-small}.

There already are theoretical arguments~\citep{baxter95a} supporting the claim
that learning an {\bf intermediate representation} shared across tasks can be
beneficial for multi-task learning. It has also already been argued~\citep{Bengio-2009}
that {\bf multiple levels of representation} can bring a benefit over a single level.

%{\large \bf Main Claim}
%\vspace*{-2mm}
\section*{Main Claim}
%\vspace*{-2mm}

We claim that deep learners, with several levels of representation, can
benefit more from self-taught learning than shallow learners (with a single
level), both in the context of the multi-task setting and from {\em
  out-of-distribution examples} in general.

%{\large \bf Contribution to Machine Learning}
%\vspace*{-2mm}
\section*{Contribution to Machine Learning}
%\vspace*{-2mm}

We show evidence for the above claim in a large-scale setting, with
a training set consisting of hundreds of millions of examples, in the
context of handwritten character recognition with 62 classes (upper-case,
lower-case, digits).

%{\large \bf Evidence to Support the Claim}
%\vspace*{-2mm}
\section*{Evidence to Support the Claim}
%\vspace*{-2mm}

In the above experimental setting, we show that {\em deep learners benefited
significantly more from the multi-task setting than a corresponding shallow
  learner}. and that they benefited more from {\em distorted (out-of-distribution) examples}
(i.e. from a distribution larger than the one from which test examples come from).

In addition, we show that they {\em beat previously published results} on this task 
(the MNIST special database 19)
and {\bf reach human-level performance} on both handwritten digit classification and
62-class handwritten character recognition.

\newpage

{\small
\bibliography{strings,strings-short,strings-shorter,ift6266_ml,specials,aigaion-shorter}
%\bibliographystyle{plainnat}
\bibliographystyle{unsrtnat}
%\bibliographystyle{apalike}
}


\end{document}
